\section{Постановка задачи}

Разработайте в MS Visual Studio программное решение на языке Си, которое реализует динамическую структуру данных (контейнер) типа <<Линейный список>>. 
Каждый элемент контейнера содержит строки символов произвольной длины. 

В программном решении следует реализовать следующие операции над контейнером: 
\begin{itemize}
\item создание и уничтожение контейнера; 
\item  добавление и извлечение элементов контейнера; 
\item  обход всех элементов контейнера (итератор); 
\item удаление из контейнера дублирующих элементов; 
\item  вычисление количества элементов в контейнере; 
\item  реверс контейнера (первый элемент контейнера становится последним, второй элемент становится предпоследним и т.д.); 
\item  объединение, пересечение и вычитание контейнеров; 
\item  сохранение контейнера в дисковом файле и восстановление контейнера из файла. 
\end{itemize}

\paragraph{Ограничения.}

Реализуйте простейший проект типа <<приложение командной строки>> (т.е. без оконного интерфейса). 
Средства C++ (объекты, классы, шаблоны классов) использовать не следует. 
Готовые контейнерные классы из библиотеки STL также использовать не следует. Разработайте контейнер самостоятельно на языке Си. 

\paragraph{Рекомендации.}
 Начните работу с изучения wiki:

 \href{https://github.com/djbelyak/OOPLab-List/wiki}{https://github.com/djbelyak/OOPLab-List/wiki}

Найдите и изучите в рекомендованной литературе и в документации MS Visual Studio описания и примеры реализаций данной структуры данных. 
Обдумайте и обсудите с преподавателем алгоритмы, состав функций, интерфейс и общую структуру программы. 
Возникающие затруднения пытайтесь преодолеть самостоятельно, потом обращайтесь за помощью. 

Письменный отчет по работе должен содержать следующие разделы: 
\begin{enumerate}
\item Постановку задачи. 

\item Описание контейнера как динамической структуры данных, в том числе: 
\begin{itemize}
	\item рисунки, на которых изображена структура данных и поясняются основные алгоритмы;
	\item описание алгоритмов, которые используются при работе с контейнером; 
	\item область применения данной структуры данных, её преимущества и недостатки. 
\end{itemize}
\item  Листинг разработанного авторского кода на языке Cи. 
Код должен быть надлежащим образом структурирован и снабжен комментариями. 
\end{enumerate}

Для успешной сдачи лабораторной работы необходимо представить письменный отчет, продемонстрировать на практике работоспособность программного решения и ответить на вопросы преподавателя. 

